
\footnote{Partly to address this problem, \citet{melcangiStock} construct a model that accounts for high saving rates among high-income households during normal times and transitorily high consumption during episodes where the infrequent consumption good becomes available (such as high-end health care, education expenses or bequests).
  
$^{.}$\footnote{One attractive feature of the splurge assumption is that it is also consistent with evidence from \cite{ganongConsumer2019}, that spending drops sharply following the large and predictable drop in income after the exhaustion of unemployment benefits; see below.}

