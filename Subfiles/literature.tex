% -*- mode: LaTeX; TeX-PDF-mode: t; -*- # Tell emacs the file type (for syntax)
% -*- mode: LaTeX; TeX-PDF-mode: t; -*- 
% LaTeX path to the root directory of the current project
% from the directory in which this file resides
% and path to econtexPaths which defines the rest of the paths like \FigDir
\providecommand{\econtexRoot}{}\renewcommand{\econtexRoot}{.}
\providecommand{\econtexPaths}{}\renewcommand{\econtexPaths}{econtexPaths}
% -*- mode: LaTeX; TeX-PDF-mode: t; -*- 
% The \commands below are required to allow sharing of the same base code via Github between TeXLive on a local machine and Overleaf (which is a proxy for "a standard distribution of LaTeX").  This is an ugly solution to the requirement that custom LaTeX packages be accessible, and that Overleaf prohibits symbolic links
\providecommand{\packages}{\econtexRoot/Resources/texmf-local/tex/latex}
\providecommand{\econtex}{\packages/econtex}
\providecommand{\econark}{\econtexRoot/Resources/texmf-local/tex/latex/econark}
\providecommand{\econtexSetup}{\econtexRoot/Resources/texmf-local/tex/latex/econtexSetup}
\providecommand{\econarkSetup}{\econtexRoot/Resources/texmf-local/tex/latex/econarkSetup}
\providecommand{\econtexShortcuts}{\econtexRoot/Resources/texmf-local/tex/latex/econtexShortcuts}
\providecommand{\econtexBibMake}{\econtexRoot/Resources/texmf-local/tex/latex/econtexBibMake}
\providecommand{\econtexBibStyle}{\econtexRoot/Resources/texmf-local/bibtex/bst/econtex}
\providecommand{\econtexBib}{economics}
\providecommand{\notes}{\econtexRoot/Resources/texmf-local/tex/latex/handout}
\providecommand{\handoutSetup}{\econtexRoot/Resources/texmf-local/tex/latex/handoutSetup}
\providecommand{\handoutShortcuts}{\econtexRoot/Resources/texmf-local/tex/latex/handoutShortcuts}
\providecommand{\handoutBibMake}{\econtexRoot/Resources/texmf-local/tex/latex/handoutBibMake}
\providecommand{\handoutBibStyle}{\econtexRoot/Resources/texmf-local/bibtex/bst/handout}

\providecommand{\FigDir}{\econtexRoot/Figures}
\providecommand{\CodeDir}{\econtexRoot/Code}
\providecommand{\DataDir}{\econtexRoot/Data}
\providecommand{\SlideDir}{\econtexRoot/Slides}
\providecommand{\TableDir}{\econtexRoot/Tables}
\providecommand{\ApndxDir}{\econtexRoot/Appendices}

\providecommand{\ResourcesDir}{\econtexRoot/Resources}
\providecommand{\rootFromOut}{..} % APFach back to root directory from output-directory
\providecommand{\LaTeXGenerated}{\econtexRoot/LaTeX} % Put generated files in subdirectory
\providecommand{\econtexPaths}{\econtexRoot/Resources/econtexPaths}
\providecommand{\LaTeXInputs}{\econtexRoot/Resources/LaTeXInputs}
\providecommand{\LtxDir}{LaTeX/}
\providecommand{\EqDir}{\econtexRoot/Equations} % Put generated files in subdirectory

\providecommand{\titlepagecustom}{\LaTeXInputs/titlepagecustom}


\documentclass[\econtexRoot/HAFiscal]{subfiles}
\onlyinsubfile{\externaldocument{\econtexRoot/HAFiscal}} % Get xrefs -- esp to apndx -- from main file; only works if main file has already been compiled

\begin{document}

\hypertarget{related-literature}{}\par\subsection{Related literature}
\notinsubfile{\label{sec:lit}}


Our paper is part of the growing literature using structural heterogeneous agent models to examine effects of countercyclical fiscal policies.

Because the quantitative implications of HA models depend profoundly on getting certain microeconomic details right, we begin with a brief synopsis of what we view as the relevant takeaways from the micro literature.\footnote{\cite{coenen2012effects} analyses the effects of different fiscal policies using seven different Two Agent New Keynesian (TANK) models, but such models make no attempt to match the key microeconomic facts.} %

\hypertarget{microeconomic-evidence}{}
\subsubsection{Microeconomic Evidence}
For our purposes, the single most important kind of micro evidence is on the iMPC; we explicitly target our partial equilibrium model to match the microeconomic iMPC estimates of \cite{fagereng_mpc_2021}, whose evidence is the gold standard because their millions of datapoints allow precise estimates over a long horizon (five years) and because their natural experiment is almost ideal (a lottery win is a random shock by construction).  Particularly striking is their evidence on the excess initial MPC.  Any worry that their Norwegian evidence might not apply in other countries is allayed by results in a new paper by \cite{kotsogiannisMPCs}, who use data from a Greek lottery and find that the induced extra monthly spending in the first three months after the win is triple the induced extra spending in the remaining observed months.

There are, of course, many prominent papers dating all the way back to \cite{friedman:windfalls} finding that the initial spending response to shocks is vastly greater than implied by a representative agent model (in particular a steady stream of state-of-the-art papers by Jonathan Parker and his collaborators). In much of that old literature there has been some evidence that the initial spending response was out of line with the subsequent effects (see, e.g., \cite{parker2013consumer}; \cite{broda2014economic}; \cite{jpsTax}), but data limitations usually made it difficult to sharply pin down the temporal pattern of spending (especially beyond the six month horizon).


Another striking result in \cite{fagereng_mpc_2021} was that even households with high liquid wealth exhibited high MPC's.  \cite{boehm2025fivefacts}, \cite{graham2024mental}, \cite{crawley2023MicroMacro}, and \cite{kueng2018excess} among others also provide strong evidence of high MPCs for high-liquid-wealth households.\footnote{The ``infrequent consumption good'' model of \cite{melcangiStock} has a similar flavor, but is not about MPC's.  It aims at accounting for high saving rates among high-income households during normal times and high consumption during episodes where the infrequent consumption good becomes available (such as high-end health care or education expenses).}

\hypertarget{microeconomic-theories}{}
\subsubsection{Microeconomic Theories}
A rapidly growing recent literature has used a variety of data sources to reconfirm the high initial MPC, but with an eye to providing theoretical explanations. We sketch this literature because different theories might have different implications for the spending consequences of a shock.


One possibility is that the burst of initial spending is rationalizable if the spending is on durables \citep{bcShocksStocks}.  
\cite{mankiw:durgoods} showed that in the frictionless case, spending on durable goods should be vastly more responsive to an income shock than spending on nondurables. 
It seems plausible that a model with a large number of goods that are durable at, say, the quarterly or annual frequency could explain the `excess initial MPC' as reflecting a rational marginal propensity to eXpend (MPX).\footnote{The NIPA accounts treat as `durable' those goods whose expected lifetime is 3 years or more, but at the annual (or quarterly) frequency many more goods (and even services) are arguably durable -- for example, \cite{bdTimeSeriesC} mention clothes and shoes, and \cite{hkpMemorable} argue that many services are durable at the annual frequency, which explains why people take vacations once a year.}

\cite{lmmPresentBias} combine a simplified model of durables spending with the assumption common in behavioral economics that spending decisions are influenced by ``present bias'' (people have time-inconsistent preferences).  They present a back-of-the-envelope calculation that yields a rough estimate that the ratio of initial spending on durables to the spending that would occur if all spending were nondurable is roughly three to one (not far from the ratio estimated in the Greek lottery episode studied by \cite{kotsogiannisMPCs}).


\cite{indarte2024explains} use high frequency bank account data to study spending responses to the U.S.\ 2021 stimulus, and find sustantial ``excess MPC's'' especially among low income households; like \cite{lmmPresentBias} they lean toward present bias as an explanation.\footnote{A related theoretical insight is provided by \citet{Lian2023-ca}, who shows that households anticipating their own future consumption mistakes can rationally exhibit higher current MPCs; this is because they know that any additional savings would be likely be disposed of suboptimally in the future.}

The logic of \cite{akerlof1985near} and \cite{cochrane1989sensitivity} suggests that the utility consequences of `near-rational' deviations from frictionless rational behavior is small.
In that spirit, \cite{BoutrosWindfall} and \cite{ilutEconomic} present models with with bounded rationality and costly re-optimization.  Building on this logic, \cite{ansQuickfix} argue that costs of reoptimization cause consumers to resort to simple ``quick-fix'' consumption heuristics; for small shocks, most people in their survey report that they anticipate that their MPC's would be one or zero.

The `splurge' component of our consumption model is a simple modelling device that lets the model match the empirical evidence, regardless of what the right deep explanation(s) may be. As we will show, the model with our splurge component is consistent all of what we described above as the key `takeaways' from the micro literature. 

\hypertarget{macroeconomic-models}{}
\subsubsection{Macroeconomic Models}
Turning now to the macroeconomic setting, a number of papers have addressed questions that are similar in spirit to ours.  For example, \cite{mckay2016role}, \cite{mckay2021optimal}, and \cite{phan2024welfare} have examined the role of automatic stabilizers in HA models.

But we follow much of the recent literature in treating recessions as `MIT shocks' -- unanticipated events. And the policies we examine are discretionary, which arguably makes sense as reflecting what occurs when the automatic stabilizers have not automatically prevented a recession.\footnote{
  In our model (and most others in the literature we are contributing to) consumers do not adjust their labor supply in response to stimulus policies. 
  This assumption is broadly consistent with the empirical findings in \cite{ganong2022spending} and \cite{chodorow2016limited}.
  However, the literature is conflicted on this subject; \cite{hagedorn2017impact} and \cite{hagedorn2019unemployment} argue that extensions of unemployment insurance affect both search decisions and vacancy creation leading to a rise in unemployment. 
\cite{kekre2022unemp}, on the other hand, evaluates the effect of extending unemployment insurance in the period from 2008 to 2014, and finds that this extension raised aggregate demand and implied a lower unemployment rate than without the policy. 
Finally, \cite{cohenDisemployment} conduct a meta-analysis of the literature on how unemployment benefits impact unemployment duration, and they find that the effects are modest.}

A relevant early contribution is by \cite{kaplan2014model} who build a model where agents save in both liquid and illiquid assets. %
Their model yields a substantial consumption response to a stimulus payment, since MPCs are high both for low-wealth households and for the many households (in their model) with high net worth but little liquid assets (the ``wealthy hand-to-mouth'').  (Though the subsequent literature finding high MPC's even by wealthy households with ample liquidity casts doubt on this mechanism.)

\cite{bayercoronavirus} study discretionary fiscal policies implemented after a large shock, in their case the COVID-19 pandemic. % %%
They find that targeted stimulus through an increase in unemployment benefits has a much larger effect than an untargeted policy.
In contrast, we find that untargeted stimulus checks have slightly larger spending effects than a targeted policy extending eligibility for unemployment insurance. %
The difference derives from the fact that -- in our model as in the data -- even middle- and high-liquid-wealth consumers have relatively high MPCs, which means that much more of the stimulus checks get spent quickly.

\cite{carroll2020modeling} also study the U.S.\ fiscal response to the COVID-19 pandemic, using a HA model similar in many respects to the one we study.  They predicted\footnote{``predicted'' because the paper was published long before any data on the actual response were available.} the consumption response to the 2020 U.S.\ CARES Act  that contained both an extension of unemployment benefits and a stimulus check.
They resolve the tension between obtaining a realistic MPC and fitting the distribution of liquid wealth by estimating the distribution of \textit{ex-ante} heterogeneity in discount factors that allows the model to match both kinds of data (discount heterogeneity is one of several competing mechanisms for resolving that tension discussed by \cite{kaplanMPC2022}).
But the model in that paper does not match the subsequently published evidence about the iMPC (\cite{fagereng_mpc_2021}); does not incorporate a multiplier; and does not compare the \textit{relative} effectiveness of alternative stimulus policies.

Another related paper is \cite{broer2025stimulus}, who analyze the output response to different fiscal policies in a HANK-and-SAM model similar to the one we present in our robustness exercise. 
They examine the policies in a steady state rather than a recession and only consider output (not consumption) responses. 
Unlike us, they do not calibrate their model to match the wealth distribution and the iMPCs, and they do not evaluate the policies using a welfare metric. 


One criterion to rank policies is the extent to which induced spending is ``multiplied,'' and our paper therefore relates to the vast literature discussing the size and timing of any multiplier. 
Our focus is on policies pursued in the Great Recession, a period when monetary policy was essentially fixed at the zero lower bound (ZLB). 
We therefore do not consider monetary policy responses to the policies we evaluate in our primary analysis, and our work thus relates to papers such as \cite{christiano2011government} and \cite{eggertsson2011fiscal}, who argue that fiscal multipliers are higher in such circumstances. 
\cite{hagedorn2019fiscal} present an HA model with both incomplete markets and nominal rigidities to evaluate the size of the fiscal multiplier and also find that it is higher when monetary policy is constrained.  They focus on government spending instead of transfers and are interested in the consequences of alternative options for financing that spending. 
\cite{broer2023fiscalmultipliers} also focus on fiscal multipliers for government spending and show how they differ in representative agent and HA models with different sources of nominal rigidities. 
Finally, \cite{ramey2018government} find empirical evidence that multipliers are higher when there is slack in the economy or the ZLB binds.  %%, the multipliers they find are still below one in most specifications. 
In any case, our concern in the versions of our model with multipliers is to compare the \textit{relative} size of any \textit{differences} in multiplication across the policies we consider, which should thus be roughly scalable by the absolute size of any multiplication effect, allowing a reader to scale our results by their preferred estimate of the magnitude of recessionary multipliers.

Aside from the size of spending effects (whether multiplied or not), we are interested in ranking policies in terms of their welfare consequences.  Thus, the paper relates to the recent literature on welfare comparisons in HA models. 
Both \cite{bhandari2021efficiency} and \cite{davila2022welfare} introduce ways of decomposing welfare effects. 
In the former case, these are aggregate efficiency, redistribution and insurance, while the latter further decomposes the insurance part into intra- and intertemporal components.

One part of these decompositions is tricky: Under standard calibrations like the ones we use, any change in redistribution tends to have powerful consequences on welfare.  We presume that there are real (but unmodeled) reasons that the equilibrium degree of redistribution in normal (nonrecessionary) times is much less than the model would call for.  We therefore develop a welfare measure that abstracts from any incentive for a planner to increase redistribution in the steady state (or ``normal'' times).

\hypertarget{organization}{}\par\subsection{Organization}
\notinsubfile{\label{sec:org}}

The paper is organized as follows. Section~\ref{sec:model} presents our baseline partial equilibrium model of households' consumption and saving problem as well as how we model a recession and the potential response in terms of three different consumption stimulus policies. Section~\ref{sec:parameters} describes the steps we take to parameterize the model and discusses the implications for some moments that we do not target. In section~\ref{sec:comparing} we compare the three policies implemented in a recession both in terms of their multipliers and in terms of a welfare measure that we introduce. Section~\ref{sec:hank} presents a general equilibrium HANK and SAM model where we compare the multipliers of the same three policies to the partial equilibrium results. Section~\ref{sec:conclusion} concludes, and, finally, the appendix shows results from a version of the model without splurge consumption and provides more details of the HANK and SAM model discussed in Section~\ref{sec:hank}. 




\onlyinsubfile{\bibliography{\bibfilesfound}}


%\onlyinsubfile{bibliography_blend}
%\onlyinsubfile{% Allows two (optional) supplements to hard-wired \texname.bib bibfile:
% system.bib is a default bibfile that supplies anything missing elsewhere
% Add-Refs.bib is an override bibfile that supplants anything in \texfile.bib or system.bib
\provideboolean{AddRefsExists}
\provideboolean{systemExists}
\provideboolean{BothExist}
\provideboolean{NeitherExists}
\setboolean{BothExist}{true}
\setboolean{NeitherExists}{true}

\IfFileExists{\econtexRoot/Add-Refs.bib}{
  % then
  \typeout{References in Add-Refs.bib will take precedence over those elsewhere}
  \setboolean{AddRefsExists}{true}
  \setboolean{NeitherExists}{false} % Default is true
}{
  % else
  \setboolean{AddRefsExists}{false} % No added refs exist so defaults will be used
  \setboolean{BothExist}{false}     % Default is that Add-Refs and system.bib both exist
}

% Deal with case where system.bib is found by kpsewhich
\IfFileExists{/usr/local/texlive/texmf-local/bibtex/bib/system.bib}{
  % then
  \typeout{References in system.bib will be used for items not found elsewhere}
  \setboolean{systemExists}{true}
  \setboolean{NeitherExists}{false}
}{
  % else
  \typeout{Found no system database file}
  \setboolean{systemExists}{false}
  \setboolean{BothExist}{false}
}

\ifthenelse{\boolean{showPageHead}}{ %then
  \clearpairofpagestyles % No header for references pages
  }{} % No head has been set to clear

\ifthenelse{\boolean{BothExist}}{
  % then use both
  \typeout{bibliography{\econtexRoot/Add-Refs,\econtexRoot/\texname,system}}
  \bibliography{\econtexRoot/Add-Refs,\econtexRoot/\texname,system}
  % else both do not exist
}{ % maybe neither does?
  \ifthenelse{\boolean{NeitherExists}}{
    \typeout{bibliography{\texname}}
    \bibliography{\texname}}{
    % no -- at least one exists
    \ifthenelse{\boolean{AddRefsExists}}{
      \typeout{bibliography{\econtexRoot/Add-Refs,\econtexRoot/\texname}}
      \bibliography{\econtexRoot/Add-Refs,\econtexRoot/\texname}}{
      \typeout{bibliography{\econtexRoot/\texname,system}}
      \bibliography{        \econtexRoot/\texname,system}}
  } % end of picking the one that exists
} % end of testing whether neither exists
}

\ifthenelse{\boolean{Web}}{}{
%  \onlyinsubfile{\captionsetup[figure]{list=no}}
%  \onlyinsubfile{\captionsetup[table]{list=no}}
  \end{document} \endinput
}
