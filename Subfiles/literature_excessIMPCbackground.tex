% -*- mode: LaTeX; TeX-PDF-mode: t; -*- # Tell emacs the file type (for syntax)
% -*- mode: LaTeX; TeX-PDF-mode: t; -*- 
% LaTeX path to the root directory of the current project
% from the directory in which this file resides
% and path to econtexPaths which defines the rest of the paths like \FigDir
\providecommand{\econtexRoot}{}\renewcommand{\econtexRoot}{.}
\providecommand{\econtexPaths}{}\renewcommand{\econtexPaths}{econtexPaths}
% -*- mode: LaTeX; TeX-PDF-mode: t; -*- 
% The \commands below are required to allow sharing of the same base code via Github between TeXLive on a local machine and Overleaf (which is a proxy for "a standard distribution of LaTeX").  This is an ugly solution to the requirement that custom LaTeX packages be accessible, and that Overleaf prohibits symbolic links
\providecommand{\packages}{\econtexRoot/Resources/texmf-local/tex/latex}
\providecommand{\econtex}{\packages/econtex}
\providecommand{\econark}{\econtexRoot/Resources/texmf-local/tex/latex/econark}
\providecommand{\econtexSetup}{\econtexRoot/Resources/texmf-local/tex/latex/econtexSetup}
\providecommand{\econarkSetup}{\econtexRoot/Resources/texmf-local/tex/latex/econarkSetup}
\providecommand{\econtexShortcuts}{\econtexRoot/Resources/texmf-local/tex/latex/econtexShortcuts}
\providecommand{\econtexBibMake}{\econtexRoot/Resources/texmf-local/tex/latex/econtexBibMake}
\providecommand{\econtexBibStyle}{\econtexRoot/Resources/texmf-local/bibtex/bst/econtex}
\providecommand{\econtexBib}{economics}
\providecommand{\notes}{\econtexRoot/Resources/texmf-local/tex/latex/handout}
\providecommand{\handoutSetup}{\econtexRoot/Resources/texmf-local/tex/latex/handoutSetup}
\providecommand{\handoutShortcuts}{\econtexRoot/Resources/texmf-local/tex/latex/handoutShortcuts}
\providecommand{\handoutBibMake}{\econtexRoot/Resources/texmf-local/tex/latex/handoutBibMake}
\providecommand{\handoutBibStyle}{\econtexRoot/Resources/texmf-local/bibtex/bst/handout}

\providecommand{\FigDir}{\econtexRoot/Figures}
\providecommand{\CodeDir}{\econtexRoot/Code}
\providecommand{\DataDir}{\econtexRoot/Data}
\providecommand{\SlideDir}{\econtexRoot/Slides}
\providecommand{\TableDir}{\econtexRoot/Tables}
\providecommand{\ApndxDir}{\econtexRoot/Appendices}

\providecommand{\ResourcesDir}{\econtexRoot/Resources}
\providecommand{\rootFromOut}{..} % APFach back to root directory from output-directory
\providecommand{\LaTeXGenerated}{\econtexRoot/LaTeX} % Put generated files in subdirectory
\providecommand{\econtexPaths}{\econtexRoot/Resources/econtexPaths}
\providecommand{\LaTeXInputs}{\econtexRoot/Resources/LaTeXInputs}
\providecommand{\LtxDir}{LaTeX/}
\providecommand{\EqDir}{\econtexRoot/Equations} % Put generated files in subdirectory

\providecommand{\titlepagecustom}{\LaTeXInputs/titlepagecustom}


\documentclass[\econtexRoot/HAFiscal]{subfiles}
\onlyinsubfile{\externaldocument{\econtexRoot/HAFiscal}} % Get xrefs -- esp to apndx -- from main file; only works if main file has already been compiled

\begin{document}

\hypertarget{related-literature}{}\par\subsection{Related literature}
\notinsubfile{\label{sec:lit}}

%% HT: I've put this stuff at the top not because I think it belongs at the top
%% but to make it easy for you to find so you can relocate it as you see fit
\hypertarget{excess-initial-mpc-literature}{}
\hypertarget{kotsogiannisMPCs}{}

The evidence for an excess initial MPC in \cite{fagereng_mpc_2021} is consistent with longstanding as well as recent findings in the literature.  Perhaps the closest direct comparison is to \cite{kotsogiannisMPCs}, who use data from a Greek (as opposed to Norwegian) lottery, and find that the induced extra monthly spending in the first three months is triple the induced extra spending in the remaining observed months.\footnote{The paper also interestingly finds a close connection between reported preferences (what consumers \textit{say} they will do in a hypothetical scenario) and their actual behavior when the scenario is realized.  Other recent papers using hypothetical questions such as \cite{jappelliIntertemporal} or \cite{colarietiHow} find estimates ranging from 15 to 26 percent in the first year depending on the size of windfall gains. This suggests that it might be to fruitful to ask respondents \textit{why} they expect to exhibit a high initial MPC.}  

Because they are focused primarily on measuring empirical facts, neither of these papers proposes a specific model for the excess initial MPC (nor do most earlier papers, such as \cite{parker2013consumer,jpsTax}).  But a substantial literature has recently developed that both provides further evidence of an excess initial MPC and proposes a number of competing theoretical models of the phenomenon.  

\hypertarget{laibson2022simple}{}

One strand of the theoretical literature explores the possibility that the burst of initial spending is rationalizable if the spending is on durables  (\cite{bcShocksStocks}).  \cite{mankiw:durgoods} showed that in the frictionless case, spending on durable goods should be vastly more responsive to a permanent income shock than spending on nondurables.\footnote{Much \textit{more} responsive than the aggregate data indicate.  \cite{caballeroDurable} proposed that a heterogeneous agent model might explain the `slow adjustment,' but \cite{cdSs} have argued that when uncertainty is added to the problem, durable goods expenditures again become much more volatile than in the data.}$^{,}$\footnote{The NIPA accounts treat as `durable' those goods whose expected lifetime is 3 years or more, but at the annual frequency many more things are arguably durable -- for example, \cite{hkpMemorable} argue that many services are durable at the annual frequency, which explains why people take vacations once a year.}  It seems plausible that a model in which consumers own a large number of goods that are durable at, say, the quarterly or annual frequency\footnote{\cite{bdTimeSeriesC} mention clothes and shoes as examples.} could explain the `excess initial MPC' as actually reflecting a rational marginal propensity to Expend (MPX).  

% Summary of papers: delete later
%\cite{BoutrosWindfall}: costly re-optimization of infinite horizon consumption paths. Small windfall gains will result in less consumption-smoothing

%\cite{ilutEconomic}: Bounded rationality model of imperfect reasoning in which households keep their old policy functions until a big enough shock hits. Therefore, even unconstraint (wealthier) households have high initial MPCs.

%\cite{Lian2023-ca}: Mistakes in future consumption results in high current MPCs. Various behavioral biases can cause these mistakes, such as inattention, present bias, diagnostic expectations, and near-rationality (epsilon-mistakes).

\hypertarget{indarte2024explains}{}

Alternatively, \cite{Lian2023-ca} shows in an intertemporal consumption saving model how deviations of the optimal consumption policy in the future result in high MPCs today. These \textit{mistakes} can arise due to many behavioral biases. \cite{BoutrosWindfall} and \cite{ilutEconomic} use costly re-optimization in a model with bounded rationality, whereas \cite{indarte2024explains} and \cite{lmmPresentBias} attribute the excess initial MPX to a form of ``present bias'' in which people have strongly time inconsistent preferences.

\cite{laibson2022simple} combine these two ideas in a simple model with both present bias and durables expenditures.  A back-of-the-envelope calculation yields a rough estimate that the ratio of initial spending on durables to the spending that would occur if all spending were nondurable is roughly three to one (not far from the ratio estimated in the Greek lottery episode studied by \cite{kotsogiannisMPCs}).

% HT comment 1: I am a bit unsure about including this. Aren't we saying that we have the splurge to get high MPCs for high-wealth households. Then we refer to a literature that aims to explain this, but here we're saying that the fact itself is a challenge to that literature? 
A potential challenge to these interpretations is the now-substantial literature showing that even high-wealth households seem to have a high initial MPC out of income shocks.\footnote{In addition to \cite{fagereng_mpc_2021}, \cite{boehm2025fivefacts}, \cite{graham2024mental}, \cite{crawley2023MicroMacro}, and \cite{kueng2018excess} among others all provide strong evidence of high MPCs for high-liquid-wealth households.}  If such households achieved their high wealth because they are less present-biased than others, present-bias may not be a good explanation for their high MPCs; and it also seems likely to be difficult to explain why high-liquid-wealth households would have a high propensity to buy durables out of a (relatively) small transitory shock.

% HT comment 2: A footnote about mankiw:durgoods could be relevant in the section on our welfare measure.
For our purposes, a further attraction to the idea that the initial excess MP[C/X] reflects durables expenditures is that the \cite{mankiw:durgoods} model implies that the marginal value of an extra dollar of expenditures on durables is equal to the marginal utility of an extra unit of nondurables spending. At least to first order, this justifies our choice in the welfare analysis to assume that the marginal utility of splurge spending is equal to the marginal utility of nondurables consumption.  Complications like convex adjustment costs can break this exact equality, but even in that case it seems likely that a variant of the arguments of \cite{akerlof1985near} and \cite{cochrane1989sensitivity} would hold: Small deviations of actual behavior from the behavior in an idealized model are likely to have small consequences for utility.



\onlyinsubfile{\bibliography{\bibfilesfound}}


%\onlyinsubfile{bibliography_blend}
%\onlyinsubfile{% Allows two (optional) supplements to hard-wired \texname.bib bibfile:
% system.bib is a default bibfile that supplies anything missing elsewhere
% Add-Refs.bib is an override bibfile that supplants anything in \texfile.bib or system.bib
\provideboolean{AddRefsExists}
\provideboolean{systemExists}
\provideboolean{BothExist}
\provideboolean{NeitherExists}
\setboolean{BothExist}{true}
\setboolean{NeitherExists}{true}

\IfFileExists{\econtexRoot/Add-Refs.bib}{
  % then
  \typeout{References in Add-Refs.bib will take precedence over those elsewhere}
  \setboolean{AddRefsExists}{true}
  \setboolean{NeitherExists}{false} % Default is true
}{
  % else
  \setboolean{AddRefsExists}{false} % No added refs exist so defaults will be used
  \setboolean{BothExist}{false}     % Default is that Add-Refs and system.bib both exist
}

% Deal with case where system.bib is found by kpsewhich
\IfFileExists{/usr/local/texlive/texmf-local/bibtex/bib/system.bib}{
  % then
  \typeout{References in system.bib will be used for items not found elsewhere}
  \setboolean{systemExists}{true}
  \setboolean{NeitherExists}{false}
}{
  % else
  \typeout{Found no system database file}
  \setboolean{systemExists}{false}
  \setboolean{BothExist}{false}
}

\ifthenelse{\boolean{showPageHead}}{ %then
  \clearpairofpagestyles % No header for references pages
  }{} % No head has been set to clear

\ifthenelse{\boolean{BothExist}}{
  % then use both
  \typeout{bibliography{\econtexRoot/Add-Refs,\econtexRoot/\texname,system}}
  \bibliography{\econtexRoot/Add-Refs,\econtexRoot/\texname,system}
  % else both do not exist
}{ % maybe neither does?
  \ifthenelse{\boolean{NeitherExists}}{
    \typeout{bibliography{\texname}}
    \bibliography{\texname}}{
    % no -- at least one exists
    \ifthenelse{\boolean{AddRefsExists}}{
      \typeout{bibliography{\econtexRoot/Add-Refs,\econtexRoot/\texname}}
      \bibliography{\econtexRoot/Add-Refs,\econtexRoot/\texname}}{
      \typeout{bibliography{\econtexRoot/\texname,system}}
      \bibliography{        \econtexRoot/\texname,system}}
  } % end of picking the one that exists
} % end of testing whether neither exists
}

\ifthenelse{\boolean{Web}}{}{
%  \onlyinsubfile{\captionsetup[figure]{list=no}}
%  \onlyinsubfile{\captionsetup[table]{list=no}}
  \end{document} \endinput
}

