% -*- mode: LaTeX; TeX-PDF-mode: t; -*- # Tell emacs the file type (for syntax)
% -*- mode: LaTeX; TeX-PDF-mode: t; -*- 
% LaTeX path to the root directory of the current project
% from the directory in which this file resides
% and path to econtexPaths which defines the rest of the paths like \FigDir
\providecommand{\econtexRoot}{}\renewcommand{\econtexRoot}{.}
\providecommand{\econtexPaths}{}\renewcommand{\econtexPaths}{econtexPaths}
% -*- mode: LaTeX; TeX-PDF-mode: t; -*- 
% The \commands below are required to allow sharing of the same base code via Github between TeXLive on a local machine and Overleaf (which is a proxy for "a standard distribution of LaTeX").  This is an ugly solution to the requirement that custom LaTeX packages be accessible, and that Overleaf prohibits symbolic links
\providecommand{\packages}{\econtexRoot/Resources/texmf-local/tex/latex}
\providecommand{\econtex}{\packages/econtex}
\providecommand{\econark}{\econtexRoot/Resources/texmf-local/tex/latex/econark}
\providecommand{\econtexSetup}{\econtexRoot/Resources/texmf-local/tex/latex/econtexSetup}
\providecommand{\econarkSetup}{\econtexRoot/Resources/texmf-local/tex/latex/econarkSetup}
\providecommand{\econtexShortcuts}{\econtexRoot/Resources/texmf-local/tex/latex/econtexShortcuts}
\providecommand{\econtexBibMake}{\econtexRoot/Resources/texmf-local/tex/latex/econtexBibMake}
\providecommand{\econtexBibStyle}{\econtexRoot/Resources/texmf-local/bibtex/bst/econtex}
\providecommand{\econtexBib}{economics}
\providecommand{\notes}{\econtexRoot/Resources/texmf-local/tex/latex/handout}
\providecommand{\handoutSetup}{\econtexRoot/Resources/texmf-local/tex/latex/handoutSetup}
\providecommand{\handoutShortcuts}{\econtexRoot/Resources/texmf-local/tex/latex/handoutShortcuts}
\providecommand{\handoutBibMake}{\econtexRoot/Resources/texmf-local/tex/latex/handoutBibMake}
\providecommand{\handoutBibStyle}{\econtexRoot/Resources/texmf-local/bibtex/bst/handout}

\providecommand{\FigDir}{\econtexRoot/Figures}
\providecommand{\CodeDir}{\econtexRoot/Code}
\providecommand{\DataDir}{\econtexRoot/Data}
\providecommand{\SlideDir}{\econtexRoot/Slides}
\providecommand{\TableDir}{\econtexRoot/Tables}
\providecommand{\ApndxDir}{\econtexRoot/Appendices}

\providecommand{\ResourcesDir}{\econtexRoot/Resources}
\providecommand{\rootFromOut}{..} % APFach back to root directory from output-directory
\providecommand{\LaTeXGenerated}{\econtexRoot/LaTeX} % Put generated files in subdirectory
\providecommand{\econtexPaths}{\econtexRoot/Resources/econtexPaths}
\providecommand{\LaTeXInputs}{\econtexRoot/Resources/LaTeXInputs}
\providecommand{\LtxDir}{LaTeX/}
\providecommand{\EqDir}{\econtexRoot/Equations} % Put generated files in subdirectory

\providecommand{\titlepagecustom}{\LaTeXInputs/titlepagecustom}


\documentclass[\econtexRoot/HAFiscal]{subfiles}
\onlyinsubfile{\externaldocument{\econtexRoot/HAFiscal}} % Get xrefs -- esp to apndx -- from main file; only works if main file has already been compiled

\begin{document}

\hypertarget{introduction}{}\par\section{Introduction}\notinsubfile{\label{sec:intro}}
\setcounter{page}{0}\pagenumbering{arabic}

Fiscal policies that aim to boost consumer spending in recessions have been tried in many countries in recent decades.  The nature of such policies has varied widely, perhaps because traditional macroeconomic models have not provided plausible guidance about which ones are likely to be most effective---either in reducing misery (a `welfare metric') or in increasing output (a `GDP metric').

But a new generation of macro models has shown that when microeconomic heterogeneity across consumer circumstances (wealth; income; education) is taken into account, the consequences of an income shock for consumer spending depend on a measurable object: the intertemporal marginal propensity to consume (iMPC) introduced in \cite{auclert2018IKC}.  The iMPC extends the notion of marginal propensity to consume to account for the speed at which households spend.  Fortuitously, new sources of microeconomic data, particularly from Scandinavian national registries, have recently allowed the first high-quality measurements of the iMPC (\cite{fagereng_mpc_2021}).

Even in models that can match a given measured iMPC pattern, the relative merits of alternative policies depend profoundly both on the metric (welfare or GDP) and on the quantitative structure of the rest of the model -- for example, whether multipliers exist and whether the degree of multiplication is different under different economic conditions. Here, after constructing a microeconomically credible heterogeneous agent (HA) model (see below for our credibility criteria), we examine that model's implications for how effects of stimulus policies depend on the existence, nature, and timing of any ``multipliers,'' which, following \cite{kmpHandbook2016}, we model in a clean and simple way, so that the interaction of the multiplier (if any) with the other elements of the model is reasonably easy to understand. %To ease interpretation of our results, as well as to keep the model tractable, our primary analysis is based on an aggregation of individual consumption responses.
This partial equilibrium analysis allows us to transparently incorporate the possibility that multipliers may be larger in recessions.  But we understand that a richer general equilibrium framework could introduce transmission channels absent from the partial equilibrium analysis, so we also analyze a standard HANK-and-SAM model GE model modified to embed our households' consumption responses.\footnote{The \href{https://econ-ark.org}{Econ-ARK} toolkit with which the partial equilibrium model was solved constructs the Jacobians necessary to connect a steady-state version of the model to the \href{https://github.com/shade-econ/sequence-jacobian}{SSJ Toolkit}.} 

\hypertarget{microeconomically-credible}{}
By ``microeconomically credible,'' we mean a model that can match both the cross-sectional distribution of liquid wealth (following \cite{kaplan2014model}'s definition of liquid wealth) and the entire pattern of the iMPC from \cite{fagereng_mpc_2021} (see \ref{fig:aggmpclotterywin} for their data and our model's fit to it).

\hypertarget{excess-initial-mpc}{}
Standard HA models\footnote{For example, the model in \cite{cstwMPC}.} can match both the initial distribution of liquid wealth and the pattern of spending in years 1-4 (for a shock that arrives in year 0).  But even a brief look at the figure convinces the eye that spending in the initial period when the shock arrives looks out of line with the smooth pattern of declining iMPC points in years 1-4.  The eye is not wrong: HA models that match liquid assets and the spending pattern in years 1-4 seriously underpredict the amount of immediate spending that occurs on receipt of the income shock.

We call this initial extra spending the `excess initial MPC.'  Below, we describe a substantial and longstanding literature in which the pattern of an excess initial MPC has been documented, and a vigorous recent literature confirming the fact with different datasets and proposing various potential theoretical explanations.

If multipliers are operative only in recessions (or are more powerful in recessions), a model that fails to capture the excess initial MPC might generate the wrong answers for the effectiveness of the alternative fiscal policies.

The purpose of our paper is not to weigh in on which (if any) of these models is right.  Instead we sought the simplest modeling device that would capture the empirical fact of an excess initial MPC and permit unambiguous welfare calculations.  We accomplish this by adding to the standard model something we call ``splurge'' beahavior, in which each household has a portion of income out of which they have a high MPC, and the remainder of their income is disposed of as in standard buffer-stock micro models with mildly impatient but time-consistent consumers.  Because of the evidence of high initial MPC's even among wealthy households, we assume that this splurge behavior is the same across households and independent of their liquid wealth holdings.\footnote{Proponents of the theoretical models described in our literature review in section \ref{sec:lit} choose to think of our splurge as a reduced form for a deeper explanation; we would not necessarily resist such an interpretation.}

Our resulting structural model could be used to evaluate a wide variety of consumption stimulus policies.  We examine three that have been implemented in recent recessions in the United States\ (and elsewhere): an extension of unemployment insurance (UI) benefits, a means-tested stimulus check, and a payroll tax cut.    

Our first metric of policy effectiveness is ``spending bang for the buck'': For a dollar of spending on a particular policy, how much multiplication is induced?  %Timing matters because in our model (following the empirical literature), the size of any ``consumption multiplier'' depends on the economic conditions that prevail when the extra spending occurs.  Our strategy to illuminate this point is twofold.
First, we calculate the policy-induced spending dynamics in an economy with no multiplier (and, therefore, with no multiplication-bang-for-the-buck).  We then follow \cite{kmpHandbook2016}'s approach to modeling the aggregate demand externality, in which output depends mechanically on the level of consumption relative to steady state. But in contrast to \cite{kmpHandbook2016}, the aggregate demand externality in our model is switched on only when the economy is experiencing a recession---there is no multiplication for spending that occurs after our simulated recession is over.  %A less stark assumption for example, the degree of multiplication depends on the distance of the economy from its steady state, or the endogenous time-varying multiplication that arises in a New-Keynesian model)would perhaps be more realistic but also much harder to assess clearly.

Because our model's outcomes reflect the behavior of utility-maximizing consumers, we can calculate another, possibly more interesting, measure of the effectiveness of alternative policies:  their effect on consumers' welfare.  Even without multiplication, a utility-based metric can justify countercyclical policy because the larger idiosyncratic shocks to income that occur during a recession may justify a greater-than-normal degree of social insurance.  We call this `welfare bang for the buck.'

The principal difference between the two metrics is that what matters for the degree of spending multiplication is how much of the policy-induced extra spending occurs during the recession (when the multiplier matters), while effectiveness in the utility metric also depends on who is doing the extra spending (because the recession hits some households much harder than others).

Because high-MPC consumers have high marginal utility, a standard aggregated welfare function would favor redistribution to such consumers even in the absence of a recession. We are interested in the degree of \textit{extra} motivation for social insurance that is present in a recession, so we construct our social welfare metric specifically to measure only the \textit{incremental} social welfare effect of alternative policies during recessions (beyond whatever redistributional logic might apply during expansions -- see section~\ref{sec:welfare}).

%Households do not prepare for our ``MIT shock'' recessions, which double the unemployment rate and the average length of unemployment spells. The end of the recession occurs as a Bernoulli process calibrated for an average recession length of six quarters, leading to a return of the unemployment rate to normal levels over time.
When the multiplier is active, any reduction in aggregate consumption below its steady-state level directly reduces aggregate productivity and thus labor income. Hence, any policy stimulating consumption will also boost incomes through this aggregate demand multiplier channel.

% We parametrize the model in two steps.  First, we estimate the extent to which consumers `splurge' when receiving an income shock. We do so using Norwegian data because it offers the best available evidence on the time profile of the marginal propensity to consume (provided by \cite{fagereng_mpc_2021}). Next we move on to the calibration of the full model on US data taking the splurge-factor as given. In the model, consumers are \textit{ex-ante} heterogeneous: The population consists of types that differ according to their level of education (which affects measured facts about permanent income and income dynamics), and their pure time-discount factors, whose distribution is estimated separately for each education group to match the liquid wealth distribution within that group. In addition, agents experience different histories of idiosyncratic income shocks and periods of unemployment, so that within each type there is \textit{ex-post} heterogeneity induced by different shock realizations. 

Our results are intuitive.  In the economy with no recession multiplier, the benefit of a sustained payroll tax cut is negligible.\footnote{One reason there is any (welfare) benefit at all, even for people who have not experienced an unemployment spell, is that the heightened risk of unemployment during a recession increases the marginal value of current income because it helps them build extra precautionary reserves to buffer against the extra risk.  A second benefit is that, for someone who becomes unemployed some time into the recession, the temporary tax reduction will have allowed them to accumulate a larger buffer to sustain them during unemployment.  Finally, in a recession, there are more people who will have experienced a spell of unemployment, and the larger population of beneficiaries means that the consequences of the prior mechanism will be greater.  But, quantitatively, all of these effects are small.}
When a multiplier exists, the tax cut has more benefits, especially if the recession continues long enough that most of the spending induced by the tax cut happens while the economy is still in recess
ion (and the multiplier still is in force).  The typical recession, however, ends long before our `sustained' wage tax cut is reversed---and even longer before lower-MPC consumers have spent down most of their extra after-tax income. Accordingly, even in an economy with a multiplier that is powerful during recessions, much of the wage tax cut's effect on consumption occurs when any multiplier that might have existed in a recession is no longer operative.

Even leaving aside any multiplier effects, the stimulus checks improve welfare more than the wage tax cut, because at least a portion of such checks go to unemployed people who have both high MPCs and high marginal utilities (while wage tax cuts, by definition, go only to persons who are employed and earning wages). The greatest ``welfare bang for the buck'' comes from the UI insurance extension, because many of the recipients are in circumstances in which they have a much higher marginal utility than they would have had in the absence of the recession, whether or not the multiplier aggregate demand externality exists.

And, in contrast to the wage-tax cut, both the UI extension and the stimulus checks concentrate most of the marginal increment to consumption at times when the multiplier (if it exists) is still powerful.  A disadvantage of the UI extension, in terms of ``spending bang for the buck,'' is that (relative to the assumed-to-be-immediate-upon-recession checks) it takes somewhat more time until the transfers reach the beneficiaries. Countering this disadvantage is the fact that the MPC of UI recipients is higher than that of stimulus check recipients, and, furthermore, the insurance nature of the UI payments reduces the precautionary saving motive. In the end, our model says that these two forces roughly balance each other, so that the spending-bang-for-the-buck of the two policies is similar. In the welfare metric, however, there is considerable marginal value to UI recipients even if they receive some of the benefits after the recession is over (and no multiplier exists). Hence, in the welfare metric, the relative value of UI benefits is increased compared with the policy of sending stimulus checks.

We conclude that extended UI benefits should be the first weapon employed from this arsenal, as they have a greater welfare benefit than stimulus checks and a similar (multiplied) spending effect.  But a disadvantage is that the total amount of stimulus that can be accomplished with the UI extension is constrained by the fact that only a limited number of people become unemployed.  If more stimulation is called for than can be accomplished via the UI extension, checks have the advantage that their effects scale almost linearly in the size of the stimulus---see \cite{beraja2023size} for a more detailed exposition of the relation between MPC and stimulus size.  The wage tax cut is also, in principle, scalable, but its effects are smaller than those of checks because recipients have lower MPCs and marginal utility than check and UI recipients.  In the real world, a tax cut is also likely the least flexible of the three tools:  UI benefits can be further extended, and multiple rounds of checks can be sent, but multiple rounds of changes in payroll tax rates would likely be administratively and politically more difficult.

One theme of our paper is that which policies are better or worse, and by how much, depends on both the quantitative details of the policies and the quantitative modeling of the economy.  

But the tools we are using could be reasonably easily modified to evaluate a number of other policies.  For example, in the COVID-19 recession in the US, not only was the duration of UI benefits extended, but those benefits were also supplemented by substantial extra payments to every UI recipient.  We did not calibrate the model to match this particular policy, but the framework could accommodate such an analysis.


\onlyinsubfile{\bibliography{\bibfilesfound}}


%\onlyinsubfile{bibliography_blend}
%\onlyinsubfile{% Allows two (optional) supplements to hard-wired \texname.bib bibfile:
% system.bib is a default bibfile that supplies anything missing elsewhere
% Add-Refs.bib is an override bibfile that supplants anything in \texfile.bib or system.bib
\provideboolean{AddRefsExists}
\provideboolean{systemExists}
\provideboolean{BothExist}
\provideboolean{NeitherExists}
\setboolean{BothExist}{true}
\setboolean{NeitherExists}{true}

\IfFileExists{\econtexRoot/Add-Refs.bib}{
  % then
  \typeout{References in Add-Refs.bib will take precedence over those elsewhere}
  \setboolean{AddRefsExists}{true}
  \setboolean{NeitherExists}{false} % Default is true
}{
  % else
  \setboolean{AddRefsExists}{false} % No added refs exist so defaults will be used
  \setboolean{BothExist}{false}     % Default is that Add-Refs and system.bib both exist
}

% Deal with case where system.bib is found by kpsewhich
\IfFileExists{/usr/local/texlive/texmf-local/bibtex/bib/system.bib}{
  % then
  \typeout{References in system.bib will be used for items not found elsewhere}
  \setboolean{systemExists}{true}
  \setboolean{NeitherExists}{false}
}{
  % else
  \typeout{Found no system database file}
  \setboolean{systemExists}{false}
  \setboolean{BothExist}{false}
}

\ifthenelse{\boolean{showPageHead}}{ %then
  \clearpairofpagestyles % No header for references pages
  }{} % No head has been set to clear

\ifthenelse{\boolean{BothExist}}{
  % then use both
  \typeout{bibliography{\econtexRoot/Add-Refs,\econtexRoot/\texname,system}}
  \bibliography{\econtexRoot/Add-Refs,\econtexRoot/\texname,system}
  % else both do not exist
}{ % maybe neither does?
  \ifthenelse{\boolean{NeitherExists}}{
    \typeout{bibliography{\texname}}
    \bibliography{\texname}}{
    % no -- at least one exists
    \ifthenelse{\boolean{AddRefsExists}}{
      \typeout{bibliography{\econtexRoot/Add-Refs,\econtexRoot/\texname}}
      \bibliography{\econtexRoot/Add-Refs,\econtexRoot/\texname}}{
      \typeout{bibliography{\econtexRoot/\texname,system}}
      \bibliography{        \econtexRoot/\texname,system}}
  } % end of picking the one that exists
} % end of testing whether neither exists
}

\ifthenelse{\boolean{Web}}{}{
%  \onlyinsubfile{\captionsetup[figure]{list=no}}
%  \onlyinsubfile{\captionsetup[table]{list=no}}
  \end{document} \endinput
}

\endinput

\hypertarget{related-literature}{}\par\subsection{Related literature}
\notinsubfile{\label{sec:lit}}

Several papers have looked at fiscal policies that have been implemented in the U.S.\ through the lens of a structural model.
\cite{coenen2012effects} analyses the effects of different fiscal policies using seven different models.
The models are variants of two-agent heterogeneous agent models and make no attempt to match the full distribution of liquid wealth as we do in this paper.
We also attempt to match the microdata on household consumption behavior, much of which has come more recently.
More closely aligned to the methodology of our paper are \cite{mckay2016role}, \cite{mckay2021optimal}, and \cite{phan2024welfare} which look at the role of automatic stabilizers.
By contrast, we consider discretionary policies that have been invoked after a recession has begun.
Another related paper is \cite{bayercoronavirus} who studies fiscal policies implemented during the pandemic.
They find that targeted stimulus through an increase in unemployment benefits has a much larger multiplier than an untargeted policy.
In contrast, we find that untargeted stimulus checks have slightly higher multiplier effects when compared with a targeted policy extending eligibility for unemployment insurance.
Our results derive from the fact that---as in the data---even high liquid wealth consumers have relatively high MPCs in our model.

This paper is also closely related to the empirical literature that aims to estimate the effect of transitory income shocks and stimulus payments.
We particularly focus on \cite{fagereng_mpc_2021}, who use Norwegian administrative panel data with sizable lottery wins to estimate the MPC out of transitory income in that year, as well as the pattern of expenditure in the following years.
We build a model that is consistent with the patterns they identify.
Examples of the literature that followed the Great Recession in 2008 are \cite{parker2013consumer} and \cite{broda2014economic}.
These papers exploit the effectively random timing of the distribution of stimulus payments and identify a substantial consumption response.
The results indicate an MPC that is difficult to reconcile with representative agent models.

Thus, the paper relates to the literature presenting HA models that aim to be consistent with the evidence from the micro-data.
An example is \cite{kaplan2014model}, who build a model where agents save in both liquid and illiquid assets.
The model yields a substantial consumption response to a stimulus payment, since MPCs are high both for constrained, low-wealth households and for households with substantial net worth that is mainly invested in the illiquid asset (the ``wealthy hand-to-mouth'').
\cite{carroll2020modeling} present an HA model that is similar in many respects to the one we study.
Their focus is on predicting the consumption response to the 2020 U.S.\ CARES Act that contains both an extension of unemployment benefits and a stimulus check.
However, neither of these papers attempts to evaluate and rank the effectiveness of different stimulus policies, as we do.

\cite{kaplanMPC2022} discuss different mechanisms used in HA models to obtain a high MPC and the tension between that and fitting the distribution of aggregate wealth.
We use one of the mechanisms they consider, \textit{ex-ante} heterogeneity in discount factors, and build a model that delivers both high average MPCs and a distribution of liquid wealth consistent with the data.
The model allows for splurge consumption and thus also delivers substantial MPCs for high-liquid-wealth households.
This helps the model match not only the initial MPC, but also the propensity to spend out of a windfall for several periods after it is obtained.

In our model, consumers do not adjust their labor supply in response to the stimulus policies.
Our assumption is broadly consistent with the empirical findings in \cite{ganong2022spending} and \cite{chodorow2016limited}.
However, the literature is conflicted on this subject and \cite{hagedorn2017impact} and \cite{hagedorn2019unemployment} find that extensions of unemployment insurance affect both search decisions and vacancy creation leading to a rise in unemployment.
\cite{kekre2022unemp}, on the other hand, evaluates the effect of extending unemployment insurance in the period from 2008 to 2014.
He finds that this extension raised aggregate demand and implied a lower unemployment rate than without the policy.
However, he does not attempt to compare the stimulus effects of extending unemployment insurance with other policies.

One criterion to rank policies is the extent to which spending is ``multiplied,'' and our paper therefore relates to the vast literature discussing the size and timing of any multiplier.
Our focus is on policies implemented in the aftermath of the Great Recession, a period when monetary policy was essentially fixed at the zero lower bound (ZLB).
We therefore do not consider monetary policy responses to the policies we evaluate in our primary analysis, and our work thus relates to papers such as \cite{christiano2011government} and \cite{eggertsson2011fiscal}, who argue that fiscal multipliers are higher in such circumstances.
\cite{hagedorn2019fiscal} present an HA model with both incomplete markets and nominal rigidities to evaluate the size of the fiscal multiplier and also find that it is higher when monetary policy is constrained.
Unlike us, they focus on government spending instead of transfers and are interested in different options for financing that spending.
\cite{broer2023fiscalmultipliers} also focus on fiscal multipliers for government spending and show how they differ in representative agent and HA models with different sources of nominal rigidities.
\cite{ramey2018government} investigate empirically whether there is support for the model-based results that fiscal multipliers are higher in certain states.
While they find evidence that multipliers are higher when there is slack in the economy or the ZLB binds, the multipliers they find are still below one in most specifications.
In any case, we condition on policies being implemented in a recession---when, this literature argues, multipliers are higher---but it is not crucial for our purposes whether the multipliers are greater than one or not.
We are concerned with relative multipliers, and the multiplier is only one of the two criteria we use to rank policies.


The second criterion to rank policies is our measure of welfare.
Thus, the paper relates to the recent literature on welfare comparisons in HA models.
Both \cite{bhandari2021efficiency} and \cite{davila2022welfare} introduce ways of decomposing welfare effects.
In the former case, these are aggregate efficiency, redistribution and insurance, while the latter further decomposes the insurance part into intra- and intertemporal components.
These papers are related to ours, but we do not decompose the welfare effects.
Regardless of decomposition, we want to (1) use a welfare measure as an additional way of ranking policies and (2) introduce a measure that abstracts from any incentive for a planner to redistribute in the steady state (or ``normal'' times).

\hypertarget{organization}{}\par\subsection{Organization}
\notinsubfile{\label{sec:org}}

The paper is organized as follows.
Section~\ref{sec:model} presents our baseline partial equilibrium model of households' consumption and saving problem as well as how we model a recession and the potential response in terms of three different consumption stimulus policies.
Section~\ref{sec:parameters} describes the steps we take to parameterize the model and discusses the implications for some moments that we do not target.
In section~\ref{sec:comparing} we compare the three policies implemented in a recession both in terms of their multipliers and in terms of a welfare measure that we introduce.
Section~\ref{sec:hank} presents a general equilibrium HANK and SAM model where we compare the multipliers of the same three policies to the partial equilibrium results.
Section~\ref{sec:conclusion} concludes, and, finally, the appendix shows results from a version of the model without splurge consumption and provides more details of the HANK and SAM model discussed in Section~\ref{sec:hank}.



\onlyinsubfile{\bibliography{\bibfilesfound}}


%\onlyinsubfile{bibliography_blend}
%\onlyinsubfile{% Allows two (optional) supplements to hard-wired \texname.bib bibfile:
% system.bib is a default bibfile that supplies anything missing elsewhere
% Add-Refs.bib is an override bibfile that supplants anything in \texfile.bib or system.bib
\provideboolean{AddRefsExists}
\provideboolean{systemExists}
\provideboolean{BothExist}
\provideboolean{NeitherExists}
\setboolean{BothExist}{true}
\setboolean{NeitherExists}{true}

\IfFileExists{\econtexRoot/Add-Refs.bib}{
  % then
  \typeout{References in Add-Refs.bib will take precedence over those elsewhere}
  \setboolean{AddRefsExists}{true}
  \setboolean{NeitherExists}{false} % Default is true
}{
  % else
  \setboolean{AddRefsExists}{false} % No added refs exist so defaults will be used
  \setboolean{BothExist}{false}     % Default is that Add-Refs and system.bib both exist
}

% Deal with case where system.bib is found by kpsewhich
\IfFileExists{/usr/local/texlive/texmf-local/bibtex/bib/system.bib}{
  % then
  \typeout{References in system.bib will be used for items not found elsewhere}
  \setboolean{systemExists}{true}
  \setboolean{NeitherExists}{false}
}{
  % else
  \typeout{Found no system database file}
  \setboolean{systemExists}{false}
  \setboolean{BothExist}{false}
}

\ifthenelse{\boolean{showPageHead}}{ %then
  \clearpairofpagestyles % No header for references pages
  }{} % No head has been set to clear

\ifthenelse{\boolean{BothExist}}{
  % then use both
  \typeout{bibliography{\econtexRoot/Add-Refs,\econtexRoot/\texname,system}}
  \bibliography{\econtexRoot/Add-Refs,\econtexRoot/\texname,system}
  % else both do not exist
}{ % maybe neither does?
  \ifthenelse{\boolean{NeitherExists}}{
    \typeout{bibliography{\texname}}
    \bibliography{\texname}}{
    % no -- at least one exists
    \ifthenelse{\boolean{AddRefsExists}}{
      \typeout{bibliography{\econtexRoot/Add-Refs,\econtexRoot/\texname}}
      \bibliography{\econtexRoot/Add-Refs,\econtexRoot/\texname}}{
      \typeout{bibliography{\econtexRoot/\texname,system}}
      \bibliography{        \econtexRoot/\texname,system}}
  } % end of picking the one that exists
} % end of testing whether neither exists
}

\ifthenelse{\boolean{Web}}{}{
%  \onlyinsubfile{\captionsetup[figure]{list=no}}
%  \onlyinsubfile{\captionsetup[table]{list=no}}
  \end{document} \endinput
}

